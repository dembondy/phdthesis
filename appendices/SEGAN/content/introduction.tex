The power industry is experiencing a significant shift away from being based on fossil fuels towards more generation from Renewable Energy Sources (RES). The tendency is that a substantial amount of the RES are distributed energy resources (DER), which alters the traditional top-down approach in the sector, where a relatively few number of large power plants provides electric power and ancillary services. Furthermore, there is a growing electrification in the energy sector with heating and transportation being the largest energy consumers. The non-dispatchable and stochastic nature of RES and the increasing electrification of consumption, call for new sources of ancillary services. One of these sources will be demand response (DR) from small-scale entities, e.g. private consumers, whose flexibility in consumption will be harnessed by aggregators \cite{pudjianto2007virtual}. With the introduction of aggregators, as providers of ancillary services, a versatile method for ancillary service modeling and performance assessment is needed.

Currently, performance assessment of units providing services to the grid operators is done as a pass/non-pass evaluation based upon the minimum service requirements \cite{EnerginetAncillary}. While performance criteria have been formulated for specific services, e.g. load frequency control \cite{gross2001analysis} or primary frequency control \cite{eto2010use}, these focus on the overall performance of the reserves seen from a grid perspective, i.e. the criteria do not evaluate the individual performance of each service-providing unit. Furthermore, until now, the performance assessment of aggregators in academia has been ad-hoc to specific aggregator implementation, e.g. \cite{vrettos2015frequency}, or the evaluation focus has been on computational or financial performance, e.g. \cite{su2012performance,rahnama2014evaluation}. Similarly, a platform for simulation of aggregation strategy is proposed in \cite{dittawit2014demand}, but the focus is on the simulation tool, which focuses only on the demand side, and not on the process of validation. None have taken a systematic approach to generally evaluating the performance of the aggregators in terms of the contractual requirements of service delivery. 

This paper presents a novel method for modeling a generic set of active power ancillary services as well as distribution system congestion management services. Furthermore, it is shown how these models can be used for performance assessment of aggregators. The article refines the concept of a service performance assessment index and Quality of Service, both treated in previous work, and further introduces a novel way of assessing the non-delivery of a service. The non-delivery assessment is proposed for verification of the aggregator service delivery.

The rest of the article is organized as follows: background concepts for the work are presented in Sec.~\ref{sec:SEGANbackground}, the modeling method is presented in Sec.~\ref{sec:SEGANmethodology} and the service performance assessment and verification indices are presented in Sec.~\ref{sec:SEGANperformance}. The use of the service modeling and indices are shown through three case studies in Sec.~\ref{sec:SEGANcasestudies} and concluding remarks are presented in Sec.~\ref{sec:SEGANconclusion}.
