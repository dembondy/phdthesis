Why (purpose):
The current definition of ancillary services is based upon the assumption that the service is provided by traditional generators. It has been proven that some sources for demand response are capable of providing ancillary services faster (have faster ramping time) than some traditional generators, e.g. electric vehicles vs. coal power plants. It is expected that with the increase of Renewable Energy Sources, the more expensive generators will be decommissioned, thus leaving the power system with a proportionally large source of demand response compared to traditional generators. Given the current definition of the ancillary service requirements, the fast response capabilities of demand response (and some traditional generators, e.g. hydropower plants) will be underutilized. The speed within which the ancillary services can be provided by the fast ramping units benefits the power system in cases of frequency excursion, which allows for smaller reserve needs. Therefore, the requirements for ancillary services must be changed so that the system operators are able to maintain system reliability at an optimal cost.

Who (audience): System Operators (TSOs, RTOs, ISOs, DSOs)

What (content): New requirements for the ancillary services based upon service performance of the units. These requirements should allow the System Operators to optimize their ancillary service activations.

Notes from 6/8/15:
Who does the classification of parameters? 
    Who assigns the values?
Integrating legacy systems (PJM e.g.)
Touch upon market aspects, look into fitness functions, but not go all the way on market design.