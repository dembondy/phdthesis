
%Question: \emph{Who} are we going to convince of \emph{what}, and \emph{how}?\\
The requirements for ancillary services (AS) in many countries are defined, due to historical reasons, on the assumption that only generators provide ancillary services. With the increase in adoption of distributed energy resources (DERs) and controllable smart loads, as well as the emergence of schemes for utilizing consumption flexibility, such as demand response (DR), new sources for ancillary services from the demand-side are available. 
These sources posses qualities that in many cases match the performance needs of the system better than traditional generators, yet their participation in the ancillary service markets is restricted due to requirements barriers. Since there are both economic and technical benefits in exploiting these qualities, a method must be designed so that system operators can readily utilize the positive qualities of both traditional and new ancillary services sources. In this paper we propose new frequency ancillary service requirements, focused on service performance, which are source/technology independent. 

By changing the AS requirements to focus on performance rather than unit capabilities and utilizing new technologies as ancillary service providers, system operators will be able to maintain better system reliability\cite{entsoe2014demand}, and increase participation in the ancillary service markets. Furthermore, service verification and settlement will benefit those players that are able to provide better quality services.

The rest of the paper is organized as follows: section~\ref{sec:currentas} presents the current ancillary service definitions and requirements; section~\ref{sec:newas} presents the new ancillary service requirements; section~\ref{sec:ancsrvDR} presents the performance properties of different new technologies that make them suitable for frequency ancillary service provision. Section~\ref{sec:ddrascasestudy} presents a case study of the impact of the new requirements, and section~\ref{sec:ddrasconclusion} presents conclusions and thoughts for future research. 

%\begin{itemize}
%	\item Increase of fluctuating RES and aging infrastructure lead to need for AS
%	\item ``Square peg in round hole'' problem
%	\item Why:
%		\begin{enumerate}
%			\item Help the system operators get the maximum out of the properties of DR, leading to better system balance/stability/reliability, see \cite{entsoe2014demand}
%			\item Incentive new players/aggregators in the energy markets by providing fair remuneration (paid for what DR can do, not for behaving like a traditional generator)
%			\item Ease verification and settlement of DR services (related to the previous point)
%		\end{enumerate}
%\end{itemize}

