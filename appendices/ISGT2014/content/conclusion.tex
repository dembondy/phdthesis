	\section{Conclusion and Outlook}
	\label{sec:conclusion}
	Drawing inspiration from the field of Control Performance Assessment, this study proposes a performance index for the evaluation of control services for DER aggregation. The index is useful for the systematic evaluation of the adequacy of different control architectures providing ancillary services.
It was shown how the index is computed, and a case study was presented in which two different control algorithms were evaluated. The results were presented and discussed, showing that the C-MPC in this case is capable of providing a better QoS.
%Evaluation of aggregation algorithms is expected to be an important part of validation of Aggregators providing DSM. 
%	 As modelled in this paper, the index only gives a general idea of the performance of the control algorithm, but future work could include a method for differentiating the sources of high error.
	In order to do a successful evaluation of an aggregation algorithm, it is important that the QoS specifications of the future ancillary services are well defined. This is a challenge in itself since many of the ancillary services assume a production baseline, which is easy to establish in traditional generators, but proves to be difficult for small households (see e.g. \cite{Borenstein}). Research effort should be put into redefining ancillary-service requirements to suit DSM, taking into account the probabilistic nature of managing a large number of units.

 The evaluation of aggregation control algorithms is an important part of a general validation framework for Aggregators. Future work will include further development of this Aggregator validation framework, where controllers can be tested under different grid and communication network topologies, as well as a diverse set of fault scenarios.
