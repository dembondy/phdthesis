\section{Discussion}
Specific terminology has been introduced to describe the proposed method. This terminology can be mapped to that of the field of \emph{Design of Experiments}, e.g. \emph{definition of service requirements} maps to \emph{definition of inner-noise factor} and \emph{definition of test inputs} maps to \emph{definition of outer-noise factors}. Specifically, the method resembles \emph{fractional factorial methods for off-line quality control}, see e.g.\cite{oehlert2010first}. In the case study presented in Sec.~\ref{sec:casestudy}, the inner factor, or controllable variable, is kept at a single level, i.e. the same activation signal is sent to the aggregator for each run of the experiment. The two outer factors, or noise variables, were varied over a distribution dictated by the operational scenario, i.e. the availability of the portfolio was varied on seven levels and, likewise, the process noise in the house simulation models was varied on seven levels. An important contribution of this work is applying this kind of formal test procedures to the problem of aggregator validation. The field of Design of Experiments is broad, and a further revision on the topic may yield a better method proposals than the one proposed here.

In this paper we focus only on the two  uncertainty sources mentioned above, therefore the test for time responsiveness, i.e. delay in the communications systems between the aggregator and a DER is not considered. This means that the test design presented in the case study is a simplified version of what an actual aggregator validation test would require. Future research must identify the relevant variables that need to be tested under the relevant operation scenarios. 

In comparison with the traditional test method, this validation procedure must capture the capabilities of a much more complex system, and therefore relies in part on simulations. As presented in \cite{steinbrink2015challenges}, the error between the used models and reality must be quantified and taken into account for the final aggregator certification. Each block in the simulation must use validated models or software. This applies to the communication systems, the grid models and the DER models. The test architecture, e.g. the one presented in \cite{buscher2015towards}, which validates the aggregators must also be validated.

There are still several open issues that need to be investigated with regards to aggregator validation. For example, the definition of the operation scenarios was only briefly discussed, and heuristics must be developed in order to define scenarios that are effective when testing aggregators.

Aggregator validation must be an ongoing process, that should be carried out periodically or whenever the aggregator portfolio or architecture changes significantly. Furthermore, aggregators are expected to participate in different electricity markets. Due to these reasons, along with the complexity of designing appropriate simulations, we believe that the task of validating aggregators should not carried out by the system operators, but by an independent third party. 