\section{Conventional resource validation and aggregator differences}
%\hl{ We explain why system operators want resources to be validated (WHY?) }
Ancillary services are essential for the reliability of the power system. Because these services play such an important role in the safe operation of the system, it is essential that the units or entities providing a service perform according to the requirements set by the Transmission System Operator (TSO). These requirements and processes are typically specific to a particular TSO and influenced by national regulations, interconnection grid codes etc. Two examples are regulations established by Energinet.dk and PJM.
\subsection{Current requirements for prequalification}\label{sec:PSSCCconventionalvalidation}
In Denmark, Energinet.dk, the danish TSO, ensures the appropriate service performance by requiring all units participating in the ancillary service markets to provide a documentation of their capabilities and go through an approval process \cite{energinet2012ancillary}. This approval process consists of a test conducted at least three weeks prior to the service delivery date. The tests for Frequency Containment Reserve (FCR) generally involves the injection of a setpoint step into the plant's governor and the measurement of the response. The test for Automatic Frequency Restoration Reserve (A-FRR) involves the tracking of a reference signal from Energinet.dk. Currently, this procedures are not formally described.
Demand resources are expected to provide a substantial amount of the ancillary services for the Danish grid in the future. Since distribution system services are not widespread yet, the concept of unit certification is non-existent at the distribution level. 

%\hl{Here we explain the current method for resource validation. (HOW?)}
%\hl{There is abig jump form Demnark to US, maybe we can sat that Denmark has not such regulations and US is more advanced in the process...}
While Denmark is starting to open up to new sources of ancillary services and standardise its test procedures, PJM (a regional transmission operator in the United States) has a standardised prequalification procedure for regulating resources\footnote{Regulation in the US corresponds roughly to the FRR of ENTSO-E.}, which consists of three consecutive area regulation tests, where PJM Performance Compliance scores indicate how well the resource follows a simulated regulation signal. A single test lasts for 40 minutes and in order to pass it the unit must score at least 75\% in three consecutive tests  \cite{pjm2015balance}. While this rule includes services provided by multiple generators at a single site, operators of demand resources are not required to be certified but must complete an initial training module on the requirements and business rules of the Regulation and Synchronized Reserve markets \cite{pjm2015certification}. Currently demand resources are only allowed to form 25\% of the total regulation \cite{pjm2015ancillary} in PJM, and therefore their certification process is still not a large concern. 

In both systems the validation tests have two goals: to ensure the communication with the units works correctly, and to validate the known performance model of the generators. Thus, a change of configuration in the setup requires a new certification of the generator. Also, a dedicated communication and measurement infrastructure between system operator and aggregator is required. The measurements must have high sampling frequency, e.g. better than 10 mHz, and high precision, e.g. sensitive to frequency deviations of $\pm$ 10 mHz. Measurement equipment that respect these requirements is expensive.
%\hl{We show in which way the current method is not applicable to aggregators and other distributed / multi-resource service providers, and conclude that an alternative method is needed (WHAT?, problem statement)}
\subsection{Problems applying current validation methods to aggregators}
The tests outlined above are specific to each system operator, but follow similar paradigms. 
The conventional test processes cannot be directly applied to portfolios of aggregated resources, mainly because a common assumption in the process is that the service delivery is performed by a single or small number of units. This allows inference of the unit's ramp capabilities through a response test, based on a known model. Also, a limited amount of precise and expensive measurement equipment needs to be installed.

An aggregator and the portfolio of units under its control behave fundamentally different from large generation units:
\begin{enumerate}
\item Individual generator units are well understood and models describing their static and dynamic properties are readily available. This is not the case for portfolios of aggregated units which are typically heterogeneous and can only be modelled through their statistical properties. This is aggravated by the fact that unit portfolios may be dynamically reconfigured during operation.
\item There is no direct equivalent to a single point of measurement: An aggregator's portfolio may consist of geographically dispersed units. Their aggregate power profile does not correspond to a measurement at any single point of the grid. Coupled to this, it is economically infeasible to install the required expensive measurement equipment at each DER.
\item Aggregators, by definition, operate a distributed system (both in control and geographical terms) in which each unit has its own response properties and requirements. This leads to an aggregated response that behaves differently from that of conventional generators.
\item Reliability concepts for distributed systems are different; specifically, the failure modes are not the same. If a component of a monolithic generator unit fails, the whole unit may have to shut down. The failure of a single unit in an aggregator portfolio will often have a minor or negligible impact on the overall performance. In a large portfolio it will usually be possible to recruit an equivalent replacement unit providing the same services as the failed one.
%\item There are many different ICT and control architectures for aggregators, and interoperability standards must be used for the internal workings of the aggregator. \hl{and whats the point?}
%\item The primary purpose of units composing an aggregator is to serve a specific energy-dependent need of the unit owner, not to provide the flexibility service, hence not all units in an aggregator portfolio may always be available.\hl{This is a property of e.g. demand response, not of aggregation. It applies to a single DR unit as well. I'd skip this argument.}
\end{enumerate}

For the above reasons, the same validation and service requirements cannot be applied to aggregators. This paper focuses on reinterpreting the validation tests to aggregators by adapting concepts from statistical testing to the power systems domain. 



%is not as straightforward for aggregated systems due to their geographical distribution, dynamic nature, complex ownership structure, and possible difference between the service requirements of individual units and the aggregate as a whole.
%The service requirements may also differ between demand response and traditional ancillary services, which leads to different validation requirements.\textcolor{red}{better?}
