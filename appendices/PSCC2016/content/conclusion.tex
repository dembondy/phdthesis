\section{Conclusion}
This work presents an initial approach to establishing a method for designing aggregator validation tests. This method differs from the traditional generator certification tests in that it relies on a statistical approach. Specifically, it reinterprets the generator certification tests to aggregators by adapting concepts from statistical testing to the problem. The validation test must be carried out with the aid of simulations, so that the stochasticity of the real world disturbances affecting the aggregator can be taken into account. 

While several of the concepts that form the proposed validation procedure, e.g. software framework for aggregator tests and aggregator performance assessment, have been addressed before, this work describes how these concepts can be unified in order to do a systematic testing of aggregators.

The validation procedure was shown through a simplified case study on an existing aggregator design. While the example shows a fictive setup, it appropriately represents the procedure.

An important step for the development of the validation method is the implementation of a complete test architecture with validated component models. With such a simulation framework, with realistic communication and DER models, communication delays can be implemented in order to test aggregators for time responsiveness. 

We consider the work presented here an important element of enabling aggregators in the smart grid, thus enabling consumption to actively participate in the secure operation of the power system. This will help the integration of renewable energy sources into the power system.
