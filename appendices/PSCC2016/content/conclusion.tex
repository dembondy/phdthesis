\section{Conclusion and Future Work}
This work presents an initial approach to establishing a methodology for designing aggregator validation tests. This method differs from the traditional generator certification tests in that it must be carried out in simulations, so that the stochasticity of the real world disturbances affecting the aggregator can be taken into account. A drawback of this method is that it relies on the accuracy and complexity of the simulation models. This means that the components of the validation tests must be validated against reality. The test method was shown through a simplified case study on an existing aggregator. While the example shows a fictive TSO applying the test to a fictive aggregator, there is the possibility that validation of aggregators in the future will be carried out by third party test companies. 

There are still several open issues that need to be investigated with regards to aggregator validation. For example, the definition of the operation scenarios was only briefly discussed, and heuristics must be developed in order to define scenarios that are effective when testing aggregators.

An important step for the development of the validation method is the implementation of a complete test architecture with validated component models. With such a simulation framework, with realistic communication and DER models, communication delays can be implemented in order to test aggregators for time responsiveness. 

Finally, the method should be expanded to cover other ancillary services, such as voltage regulation.

We consider the work presented here an important element of enabling aggregators in the smart grid, thus enabling consumption to actively participate in the secure operation of the power system. This will help the integration of renewable energy sources into the power system.
