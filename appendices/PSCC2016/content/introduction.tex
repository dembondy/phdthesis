\section{Introduction}
As renewable energy generation increasingly replaces conventional power plants, power system operators are looking for alternative sources for the ancillary services which were traditionally provided by these plants. It is expected that some ancillary services can be procured from distributed consumption and production units by making use of their unused operational flexibility. In order to provide coordinated services, such as demand response \cite{macdonald2012demand}, from a large number of such distributed energy resources (DERs), a new actor is appearing in the power system: the aggregator \cite{gkatzikis2013a}.% \hl{I was wondering whether we should clarify which type of aggregator we're talking about here, to make it harder for nitpicking reviewers.}\textcolor{red}{You mean commercial VPP vs Technical VPP?}

Conventional sources for ancillary services must be certified before being able to offer control reserves in an ancillary services market \cite{energinet2012ancillary}. It is expected that aggregators will be required to undergo a similar prequalification process to ensure the appropriate performance of the provided service with respect to predefined requirements.

The achievable performance of aggregators, in terms of service provision, depends on its architecture \cite{bondy2015a}, i.e. where the decision making is located \cite{kosek2013overview} and its level of automation, the choice of hardware for implementation, the specification of communication protocols \cite{kiliccote2010open}, how advanced the portfolio management is, etc. This means that some aggregator architectures will be better suited for a specific ancillary service than others \cite{bondy2014performance}. 

It has been established that current requirements for participation in the ancillary services markets limit the participation of aggregators of demand response \cite{cappers2013assessment,coalition2014mapping}. Different research projects have looked into this problem, see e.g. \cite{bondy2014flech}. 

Until now, the performance evaluation and testing of aggregators in academia has been ad-hoc to specific aggregator implementations \cite{vrettos2015integrating,rahnama2014evaluation}. Aggregator test frameworks have been proposed \cite{buscher2015towards}, field tests have been carried out in order to validate DR schemes \cite{kiliccote2013field}, and concepts regarding systematized testing have been exemplified \cite{steinbrink2015challenges}.

This work addresses the gap between all these concepts, i.e. we present a procedure for the design of validation tests, which takes a systematical approach to aggregator testing, discussing the issue from input, i.e. service requirements, to output, i.e. performance metrics.

%This work addresses the gap between all these concepts, i.e. the design of a test procedure that takes a systematical approach to aggregator testing, discussing the issue from input, i.e. service requirements, to output, i.e. performance metrics.

%Therefore new methods and tools are needed to validate an aggregator architecture, in terms of capability to deliver the desired services. %Such an assessment of an aggregator can be relevant before investments into a particular infrastructure are made. The methods may also be incorporated into an aggregator prequalification process.

%The purpose of this paper is to derive the testing specification for an aggregator from its contracted service requirements. Specifically, we constrain the analysis to active power services such as frequency containment and frequency restoration reserves \cite{entso1operational} for transmission systems and congestion management for distribution systems. The novelty presented in this work is a method for designing validation tests for aggregators. \hl{How can I give this more oomph?!}

%of the interaction between the systems interacting directly or indirectly with the aggregator. \textcolor{red}{The word interaction is used too much here, how do I fix this?}\hl{I'm not sure if I know what you want to say. "testing of the interaction between the systems interacting with the aggregator" - what interacts with what here? My best bet right now is "The purpose of this paper is to describe the alignment of service requirements with the testing specification of the aggregator, including the interaction between its subsystems." Is that what you meant?}

