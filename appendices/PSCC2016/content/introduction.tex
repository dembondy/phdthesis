\section{Introduction}
As renewable energy generation increasingly replaces conventional power plants, power system operators are looking for alternative sources for the ancillary services which were traditionally provided by these plants. It is expected that some ancillary services can be procured from distributed consumption and production units by making use of their unused operational flexibility. In order to provide coordinated services, such as demand response, from a large number of such distributed energy resources (DERs), a new actor is appearing in the power system: the aggregator \cite{gkatzikis2013a}.% \hl{I was wondering whether we should clarify which type of aggregator we're talking about here, to make it harder for nitpicking reviewers.}\textcolor{red}{You mean commercial VPP vs Technical VPP?}

Conventional sources for ancillary services must be %certified/
validated before being able to offer control reserves in an ancillary services market \cite{energinettender}. It is expected that aggregators will be required to undergo a similar validation process to ensure the integrity of the provided service with respect to predefined requirements.

The cost of establishing an aggregator is driven partly by the cost of establishing the associated control and communication infrastructure. The communication infrastructure requirements and installation effort are strongly dependent on the chosen aggregator architecture \cite{kosek2013overview}. The achievable performance with a given concept also depends on this architecture. %\hl{Do we have a reference for that statement?}\textcolor{red}{I don't think so, but isn't logical? Direct and indirect control will have very different infrastructure implementations... I think rather I'm not clear in what I'm trying to express} 
Therefore new methods and tools are needed to validate an aggregator architecture, in terms of capability to deliver the desired services. Such an assessment of an aggregator can be relevant before investments into a particular infrastructure are made. The methods may also be incorporated into a future aggregator validation process.

The purpose of this paper is to derive the testing specification for an aggregator from its contracted service requirements. Specifically, we constrain the analysis to active power services such as frequency containment and frequency restoration reserves \cite{entso1operational} for transmission systems and congestion management for distribution systems. %of the interaction between the systems interacting directly or indirectly with the aggregator. \textcolor{red}{The word interaction is used too much here, how do I fix this?}\hl{I'm not sure if I know what you want to say. "testing of the interaction between the systems interacting with the aggregator" - what interacts with what here? My best bet right now is "The purpose of this paper is to describe the alignment of service requirements with the testing specification of the aggregator, including the interaction between its subsystems." Is that what you meant?}
