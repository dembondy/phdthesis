%\khnote{to be Reviewed}Traditional sources for ancillary services are diminishing as intermittent distributed renewable energy sources supplant traditional fossil-fueled generation units. 
Aggregation of large quantities of consumption units is expected to be a new source of power system ancillary services. For large and conventional   generation units the dynamic response is well understood and detailed individual measurement is feasible, which factors in to the straightforward performance requirements applied today. Aggregation-based ancillary service delivery can be very responsive and fast, but the dynamic response can also be uncertain, subject to both variations in aggregator infrastructure and algorithms as well as diversity of flexibility resources. For secure power system operation, a reliable service delivery is required, yet it may not be appropriate to apply conventional performance requirements. %and mental generator equivalents directly down to individual units of an aggregator portfolio. 
The service performance requirements and assessment method therefore need to be adapted to service provision from aggregators.  
%As the actual point of delivery of an aggregate response, is virtual rather than a single physical connection point,  
This paper develops a modeling method for ancillary services performance requirements applicable to aggregation portfolios, %The service models are useful for assessing the performance of aggregators.
including performance and verification indices. % for aggregator service provision.
The use of the modeling method and the indices is exemplified in two case studies.
%These concepts are critical if aggregators are to help ensure the security of the power grid in a future with high amount of intermittent power generation. 