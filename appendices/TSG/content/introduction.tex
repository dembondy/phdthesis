The power industry is experiencing a significant shift away from being based on fossil fuels towards more generation from Renewable Energy Sources (RES). The tendency is a substantial increase in the amount of RES, often as distributed energy resources (DER), and a growing electrification of the heating and transportation sectors \cite{iea2012a,eurelectric2011}. The non-dispatchable and stochastic nature of RES and the increasing electrification of consumption call for new sources of ancillary services, as conventional generation is pushed out of the market. This alters the traditional distribution of flexibility resources in the sector, where relatively few large power plants provides electric power and ancillary services (AS). A new AS resources will be demand response (DR) from small-scale entities, such as commercial buildings or private households, whose flexibility in consumption will be harnessed by aggregators \cite{pudjianto2007virtual,koch2011modeling,biegel2013primary,vrettos2015integrating,mathieu2012using,sullivan2013using}. 
With the introduction of aggregators as providers of ancillary services, the AS specifications are being adapted to new resource types, but also prequalification and verification of service delivery need to be adapted to be suitable for the aggregated service delivery \cite{bode2013incorporating}. This is relevant both due to the change in ancillary service specifications and due to the introduction of new distribution system services \cite{heussen2013clearinghouse}.

Currently, the verification of ancillary service delivery typically is based on a rigid performance assessment (pass/non-pass) of the units providing services \cite{EnerginetAncillary}. Based upon the FERC order 755 \cite{order755}, PJM, an American regional transmission operator, has implemented a pay-for-performance scheme by evaluating the performance of frequency regulation units, thus changing the rigid verification procedures.
%to the grid operators is done as a pass/non-pass evaluation based upon the minimum service requirements \cite{EnerginetAncillary}\kh{that's not the verification - that's validation / prequalification; focus toward verification}\bondy{This is also valid for the verification: either you deliver or you don't, there's no spectrum}\kh{then reformulate - state in relative terms, clarifying relationship btw. verification  and perfomance assessement, also accounting for PJMs model (i.e. service verification is commonly based on a rigid performance assessment (pass/non-pass); recent developments ... PJM}. 
% idea: mark (in comments like here), what the different research fields you compare to are
While performance criteria have been formulated for specific services, e.g. load frequency control \cite{gross2001analysis} or primary frequency control \cite{eto2010use}, these focus on the overall performance of the reserves seen from a grid perspective, i.e. the criteria do not evaluate the individual performance of each service-providing unit. PJM has introduced a performance score which evaluates the unit performance, but its definition is tied to the regulation (reference tracking) service \cite{pjm2015balance}.

It is clear from the literature that the interpretation of performance assesment of aggregators varies widely. This is examplified by the ad-hoc evaluations of specific aggregator implementations, e.g. \cite{vrettos2015integrating}, or the when the performance evaluation focuses on non-service related metrics like computational or financial performance, e.g. \cite{su2012performance,rahnama2014evaluation}. This inconsistency in performance evaluation is a consequence of the lack of clear service requirements for aggregators. If aggregator are to deliver ancillary services, it must be clear on what grounds they are being verified. This issue must be addressed through the formulation of standard performance requirement models.

This paper presents a method for modeling a generic set of active power ancillary services requirements, as well as requirements for distribution system congestion management services. Furthermore, it is shown how these models can be used for performance assessment of aggregators. The article refines the concept of a service performance assessment index and Quality of Service, both treated in previous work, and further introduces a new metric for assessing the non-delivery of a service. The non-delivery assessment is proposed for verification of the aggregator service delivery.

The rest of the article is organized as follows: background on AS verification and changing characteristics are presented in Sec.~\ref{sec:TSGbackground}, the modeling method is presented in Sec.~\ref{sec:TSGmethodology} and the service performance assessment and verification indices are presented in Sec.~\ref{sec:TSGperformance}. The use of the service modeling and indices are shown through two case studies in Sec.~\ref{sec:TSGcasestudies} and concluding remarks are presented in Sec.~\ref{sec:TSGconclusion}.
