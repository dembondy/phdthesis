This section presents the requirements for the service models, as well as a method for deriving these models.

\subsection{Requirements of service models}
Through analysis of the TSO services defined in \cite{EnerginetAncillary}, the potential DSO services defined in \cite{ding2013development} and asset management services, a set of requirements for the models have been established:
\begin{itemize}
    \item[M-R1] the model must clearly identify the SLOs of the service,
    \item[M-R2] the model must incorporate both the ideal and acceptable service provision in a measurable/quantifiable way, i.e. performance metrics must be able to be applied to it,
    \item[M-R3] the models must be technology agnostic,
    \item[M-R4] since flexibility services imply a change of consumption pattern over a period of time, the models must consist of time series
\end{itemize}

Furthermore, based upon previous work of the authors \cite{bondy2014performance}, the following requirements are defined for the performance metrics:
\begin{itemize}
	\item[P-R1] provide a \emph{quality} measure normalized with respect to the contractual requirements (bounds) of a service and with respect to time,
	\item[P-R2] provide a \emph{reliability} measure in relation to service non-delivery, which is normalized with respect to time,
	\item[P-R3] service quality and reliability evaluation must be applicable to entities providing multiple services.
\end{itemize}


%\kh{just listing a few items; what comes to mind: mention all aspects: service model, performance assessment/verification, measurement options (do you know of Kempton's 'red box'? - certified 'measurement' placed at aggregator ) http://www.udel.edu/V2G/resources/test-v2g-in-pjm-jan09.pdf }



%R4: quantifiable reliability (/failure rates); soft reliability constraints (vs. N-1 type penalty)
%R4x: progressive penalization
%Rx:
%Having defined the method for formulating the service models, we will show how these can be used for


\subsection{Method for formulation of requirements model}\label{susec:reqmodform}


%The lack of a formal method for transforming contractual requirements into a mathematical model (requirements model) that can be used for automated verification of service delivery. 

%This requirements model forms the base-line against which the delivered service is benchmarked. 
%The method presented in this section outlines the steps to transforming contractual requirements into a requirements model.
Based upon the requirements [M-R1..M-R4], the method for SLA modeling is defined by the following six steps:
%form have been identified as a generic method for modeling ancillary services. The steps are exemplified by DSO services defined in  and TSO services defined in :
\begin{enumerate}
  \item Identify physical parameters defining the service [M-R1],
%  \begin{itemize}
%    \item For example: Power production or consumption, measured grid frequency, time measurements. Including maximum measuring sensitivities.
%  \end{itemize}
  \item Identify the dynamic behaviors of the service related to system parameters (if any) [M-R1],
%  \begin{itemize}
%    \item Example: Primary frequency regulation comes with a linear behavior between grid frequency and generator set-point. Power-cap has a dynamic relationship between feeder load and the controllable load power in order to keep the total feeder load at a $P_{DSO,Ref}$ value.
%    \item PowerMax is not dynamic. The aggregator must control $\Delta P_{Agg}$ to ensure that he does not violate $P_{max,Agg}$. But the service does not require a dynamic behavior related to a system parameter like primary frequency regulation and PowerCap.
%  \end{itemize}
  \item Identify the physical size of the service and the tolerated error [M-R2], % Both ideal service and minimum required service.
%  \begin{itemize}
%    \item Physical size is for example $P_{max,Agg}$ for PowerMax service or regulation bid for primary frequency regulation.
%    \item Tolerance is for example $P_{max,Agg}+P_{tolerance}$ for the PowerMax service and allowed dead-band for DK1 primary reserve.
%  \end{itemize}
  \item Identify the ideal response time of the service and acceptable response [M-R2]
%  \begin{itemize}
%    \item Most contracts comes with some timing specifying how fast the service provider must act.
%    \item E.g. DK1 primary reserve must provide 50 \% of service within 15 s and 100 \% within 30 s.
%    \item The ideal service is for example an instantaneous step in power to 100 \% of set-point for DK1 primary reserve.
%  \end{itemize}
  \item Based on the dynamics, size and timing of the service, as well as the tolerated errors from points 1--4, develop a time series for ideal and acceptable service provision. The model will be a set of time series: $\mathbf{x}_{ideal}(t)$ for ideal response and $\mathbf{x}_{acc}(t)$ for acceptable response. Both time series can be a scalar or a vector, e.g. $\mathbf{x}_{acc}(t)$ can be formed by a set of upper and lower tolerance bounds or simply by an upper bound [M-R4],
  %§$\mathbf{x}_{ideal}(t)$ and $\mathbf{x}_{acc}(t)$ can be a pair of values, e.g. minimum and maximum tolerance limits, for some services and may be only a single value, e.g. minimum or maximum tolerance, for others. 
  \item Identify how the service error is to be measured [M-R1].
\end{enumerate}

By only defining the SLA models in terms of performance, not in specific unit capabilities, the models implicitly comply with [M-R3].


\subsection{Generic model components}
We identify three service model patterns: reference tracking, band service or a maximum/minimum cap. The error measure, $e(t) \in \mathbb{R}$, for each of the service types is defined in the following subsections. This approach was initially introduced in \cite{bondy2014performance}, and is further refined in this work. 



%The error, $e(t) \in \mathbb{R}$, in the delivery of a service can be calculated using a reference tracking, band service or a maximum/minimum cap approach. The proposed approach depends on the specific service and contract.\bondynote{These last two paragraphs can probably be merged}


\subsection*{Reference tracking}
Reference tracking error can be calculated as:
\begin{equation}\label{eq:TSGref_error}
e(t) = x_{meas}(t) - x_{ideal}(t),
\end{equation}
where $x_{meas}(t)$ is the measured load/generation and $x_{ideal}(t)$ is the signal to be tracked. This definition will lead $e<0$ for measured values below the ideal and $e>0$ for values above the ideal. In this case $\mathbf{x}_{acc}(t)$ will be a band around $x_{ideal}(t)$, and the values of $\mathbf{x}_{acc}(t)$ do not need to be symmetric.

\subsection*{Band service}
The ideal response in a band service is defined as $ \mathbf{x}_{ideal}(t)= [x_{min}(t),x_{max}(t)]$. The error in the band service can therefore be estimated by:
\begin{equation}\label{eq:TSGband_error}
e(t)=
\begin{cases}
x_{meas}(t) - x_{min}(t) , & x_{meas}(t) < x_{min}(t)  \\
0, & x_{min}(t) \leq x_{meas}(t) \leq x_{max}(t) \\
x_{meas}(t) - x_{max}(t), & x_{meas}(t)  > x_{max}(t).  
\end{cases}
\end{equation}
In this case, the $\mathbf{x}_{acc}(t) = [\mathbf{x}_{acc,min}(t),\mathbf{x}_{acc,max}(t)]$ is a set of values that surround the band defined by $ \mathbf{x}_{ideal}(t)$, as seen in Fig.~\ref{fig:RefErr}. The values of $\mathbf{x}_{acc}(t)$ do not need to be symmetric around the band. 

\subsection*{Cap service}
%Maximum/minimum cap error only counts the error when performance is above/below some ideal value. 
In cap services, error is only tracked when $x_{meas}(t)$ is either above or below a given a limit value.
Maximum cap error is calculated as shown in \eqref{eq:TSGmaxmin_cap} and minimum cap can be similarly calculated. In \eqref{eq:TSGmaxmin_cap}, $x_{max}(t)$ is the ideal maximum limit according to the service contract:

\begin{equation}\label{eq:TSGmaxmin_cap}
e(t)=
\begin{cases}
x_{meas}(t)-x_{max}(t), & x_{meas}(t) > x_{max}(t) \\
0, & x_{meas}(t) \leq x_{max}(t).
\end{cases}
\end{equation}
In the cap service, $x_{acc}(t)$ is a limit that either lies below $x_{min}(t)$ or above $x_{max}(t)$.

