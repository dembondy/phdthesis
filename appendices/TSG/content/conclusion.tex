This paper presents a method for modeling ancillary service requirements. These models can be used  for evaluating the performance of a service provision. The performance is assessed by means of a service performance assessment index and a service verification index. The use of the modeling method and the indices are illustrated with two case studies covering a traditional ancillary service and a new distribution system service. The main purpose behind the development of the modeling method and the indices is to expand the current service verification methods to be suitable for aggregated demand response. 
%provide key components in a framework for prequalification of aggregator algorithms. As the framework is developed, the usefulness of the presented concepts will be shown and reiterated upon in case shortcomings are found. 
The performance assessment of aggregators in terms of the services they are to provide is an important element in integrating new sources of ancillary services in the power system. These new sources are expected to play an important part in the security of the future power system. %The usefulness of the indexes might be further investigated and the method revised by means of laboratory tests and small scale demonstrations on units connected to the power grid.

