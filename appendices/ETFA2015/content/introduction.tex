\newchapter{T}{he} increase of electricity production from fluctuating renewable sources is creating a need for new ways of operating the power system. Demand response (DR), i.e. the exploitation of flexibility in electricity consumption, is considered a promising technology for mitigating this problem. However, a significant part of the DR potential exists in distributed, small and medium-sized loads. It is not practical for a power system operator to interact directly with all these flexibility assets. The role of aggregators is the creation and management of a portfolio of flexibility assets and  representation of this combined flexibility to a system operator and/or market.

System operators today rely on generators for ancillary services to maintain reliable system operation. Generators undergo validation tests and continuous monitoring on the generator site. With ancillary services provided by aggregators, similar validation and performance requirements will have to be established. However, validation and monitoring requirements cannot effectively be translated from single site monitoring to distributed aggregator control systems, and today's on-site monitoring cannot be scaled to distributed flexibility assets. 

We propose a functional aggregator reference architecture that facilitates specification and validation of aggregator functional requirements and the generic modeling of contractual and verification performance requirements. Application of the proposed functional architecture to  different aggregator designs suggests it as a meaningful benchmark for technology maturity.

% SUGGESTING TO REMOVE THIS PARAGRAPH FOR THE SHORT PAPER
%The paper presents a short overview of the current state of aggregation in Section~\ref{sec:agginsg}. The motivation for the reference architecture are presented in Section~\ref{sec:requirements} and an analysis of the aggregator functionality is presented in Section~\ref{sec:funcdec}. Section~\ref{sec:refarch} presents the proposed framework based upon the functionality analysis, and Section~\ref{sec:applic} shows how the framework can be applied to a set of academic and commercial aggregators.

%usual blabla about Smart Grids, and which problems aggregators are supposed to address in the Smart Grid context (scalability/divide-and-conquer, threshold to market entry, competition and indirectly robustness because of multiple implementations etc.). 
%\begin{itemize}
%\item Commercial aggregators are being developed by multiple parties and see their first field use. All of these are one-off designs.
%\item Performance evaluation and service validation will become important issues to be solved once aggregators are supposed to leave the protected field test environment and enter a competitive market.
%\item However, the wealth of different designs and solutions makes finding a common benchmark for evaluation and validation difficult.
%\end{itemize}
%This paper proposes a generic reference architecture for the performance evaluation of aggregators which can be used for ... and makes ... easier.

%Also, this work can serve as a checklist for companies that seek to start an aggregator business.
%Aggregation of large quantities of small, medium sized loads or a few large loads is a solution for harnessing resources that are useful for maintaining power quality and reliability in power systems with large penetration of fluctuating renewable energy sources.