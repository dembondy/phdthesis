\chapter{Resum\'e}
Demand response (forbrugs fleksibilitet) vil være vigtig for integrationen af vedvarende energikilder i elnettet. Ved at styre store mængder af småforbrugsenheder vil aggregatorer af demand response levere systemydelser til transmissionsnetoperatører og fleksibilitetsydelser til distributionsnetoperatører og balanceansvarlige aktører. Da disse ydelser er essentielle for forsyningssikkerheden, skal aggregatorer valideres. Den valideringsproces, der bliver brugt på traditionelle enheder, kan ikke bruges på aggregatorer, da de består af geografisk distribuerede heterogene enheder.

 Med udgangspunkt i den traditionelle metode brugt af transmissionsnetoperatører til validering af kraftværker præsenterer denne afhandling en metode for aggregatorvalidering. Metoden består af dokumentation af aggregatorens egenskaber og en konceptuel ramme for aggregatorvalideringstests.
 
Dokumentationen af aggregatoregenskaber udføres gennem en funktionel referencearkitektur til aggregatorer. Referencearkitekturen identificerer de basale funktioner, som en aggregator skal have for at kunne levere vellykkede ydelser. 
 
Den konceptuelle ramme for validering definerer opstillingen for aggregatortests. Disse tests skal fange den stokastiske natur af aggregatorer, og valideringsmetoden må derfor gøre brug af koncepter fra statistisk testdesign.
 
Inputtet til valideringstestene sker i form af standard scenarier og ydelsesbetingelser. Ydelsesbetingelserne er omformuleret til at inkludere nye teknologier. Omformuleringen sker ved at definere systemydelser på basis af ydelsespræstation og ved at fjerne kravene, der antager, at ydelserne er leveret af centrale kraftværker.
 
Outputtet af valideringstestene er evalueringen af ydelsespræstation og ydelsesverificering. Også disse koncepter er gendefineret, så de passer til aggregator konceptet. Gennem generelle ydelsesmodeller og præstationsindikatorer (inspireret af koncepter fra reguleringsteknik) kan ydelseslevering fra aggregatorer evalueres.

