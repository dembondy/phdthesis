\chapter{Resum\'e}
Demand response (forbrugs fleksibilitet) vil være vigtig for integrationen af vedvarende energikilder i elnettet. Ved at styre store mængder af småforbrugs enheder, vil aggregatorer af demand response lever systemydelser til transmissionsnet operatører og fleksibilitetsydelser til distributionsnet operatører og balanceansvarlige aktører. Da disse ydelser er essentielle for forsyningssikkerheden, skal aggregatorer valideres. Den valideringsproces der bliver brugt på traditionelle enheder kan ikke bruges på aggregatorer da består af geografisk distriburede heterogene enheder.

Med udgangspunkt i den traditionelle metode brugt af transmissionsnet operatører til validering af kraftværker, præsentere denne afhandling en metode for aggregatorvalidering. Metoden består af dokumentation af aggregatorens egenskaber og en konceptuel ramme for aggregator valideringstests.

Dokumentationen af aggregator egenskaber er gjort gennem en funktionel reference arkitektur til aggregatorer. Reference arkitekturen identificere de basale funktioner som en aggregator skal have for at kunne levere vellykket ydelser.

Den konceptuelle ramme for validering definerer opstillingen for aggregatorer tests. Disse tests skal fange den stokastiske natur af aggregatorer, og valideringsmetoden må derfor gør brug af koncepter fra statistisk test design.

Inputet til valideringstest sker i form af standard scenarier og ydelsesbetingelser. Ydelsesbetingelserne er omformuleret til at være inklusive over for nye teknologier. Omformuleringen sker ved at definere systemydelser på basis af ydelsespræstation, og ved at fjerne kravene der antager, at ydelserne er leveret af centrale kraftværker.

Outputet af valideringstest er evalueringen af ydelsespræstation og ydelsesverificering. Også disse koncepter er gendefineret til at indpasse aggregatorer. Gennem generelle ydelsesmodeller og præstations indikatorer (inspireret af koncepter fra reguleringsteknik) kan ydelselevering fra aggregatorer evalueres.
