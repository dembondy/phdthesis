% TEX root = ../Thesis.tex
\chapter{Summary}
Demand response will become important for the integration of renewable energy sources in the power system. By controlling large pools of small-sized consumption units, aggregators of demand response will provide ancillary services to Transmission System Operators and flexibility services to Distribution System Operators and Balance Responsible Parties. Since these services are essential for the secure operation of the grid, the aggregators must be validated. The process applied to traditional ancillary service resources can not be applied to aggregators since they are composed by geographically distributed heterogeneous resources.

Parting from the current methods employed by Transmission System Operators for validating ancillary service resources, this thesis presents a method for validating aggregators. The method consists in documentation of the aggregator capabilities and a conceptual framework for aggregator validation testing. 

The documentation of aggregator capabilities is done through a Functional Reference Architecture for aggregators. The reference architecture identifies the basic functions that an aggregator must posses in order to do a successful service provision. 

The conceptual validation framework defines the test setup for aggregator validation tests. These tests must be capture the stochastic nature of the aggregator, and therefore concepts form statistical test design are applied the validation procedure.

The inputs to the validation tests come in the form of benchmark scenarios and service requirements. The service requirements are redefined in order to be inclusive of new technologies as ancillary service resources. This is done by redefining services in terms of performance, and removing requirements that assume service provision by large centralized generators.

The outputs of the validation tests are the service performance evaluation and the service verification. Also these concepts are redefined to suit the aggregator. Through general service models and service performance indices (inspired by the field of Control Performance Assessment), the service provision from aggregators can be evaluated.


