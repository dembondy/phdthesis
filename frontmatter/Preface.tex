%!TEX root = ../Thesis.tex
\chapter{Preface}
This thesis was prepared at the Energy Systems Operation and Management group under the Center for Electric Power and Energy, which is part of the department of Electrical Engineering at the Technical University of Denmark. The thesis is a requirement for acquiring the Ph.D. degree in engineering, and was funded by Innovation Fund Denmark through the Strategic Platform for Innovation and Research in Intelligent Power (iPower), the Programme for Energy Technology Development and Demonstration (EUDP) through PowerLabDK and the Technical University of Denmark (DTU).

The energy sector is moving away from fossil fueled electricity production to generation from intermittent renewable energy sources. In order to achieve a successful integration of these renewable sources in the power system, the use demand response is essential. Demand response is expected to deliver ancillary services to the power system and the demand response schemes must therefore be validated. 

This thesis addresses the topic of aggregation of flexible units for verifiable demand response services in partial replacement of traditional ancillary service resources.
The contribution to this topic is a revision of the validation procedure, the reformulation of service requirements, restructuring of ancillary service products and new metrics for verification of service delivery, all to account for the characteristics of demand response.

This thesis is multidisciplinary in its approach and draws upon concepts from fields such as: Power systems engineering, Control engineering, Software engineering, Systems engineering, Energy policy and regulation.

The project was supervised by Senior Scientist Henrik W. Bindner, and co-supervised by Associated Professor Hans Henrik Niemann, and Assistant Professor Kai Heussen, all three from DTU Electrical Engineering. Part of the research was conducted at the Lawrence Berkeley National Laboratory with S{\i}la K{\i}l{\i}\c{c}\c{c}ote as supervisor.

The thesis consists of a synthesis (along with adjustments and expansions) of the concepts presented in two journal papers and three conference papers written in the period 2012-2016.
\vfill

{
\centering
%    \thesislocation{}, \today\\[1cm]
%    \hspace{3cm}\includegraphics[scale=0.4]{Signature}\\[1cm]
\begin{flushright}
    \thesisauthor{}\\
March 2016
\end{flushright}
}
