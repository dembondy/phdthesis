%!TEX root = ../Thesis.tex
\chapter{Acknowledgements}
%I started the work on my PhD project in autumn 2012, and through these three years I have collaborated with many inspiring people, too many to write all their names here. I am thankful to my colleagues at DTU for being a source of inspiration, especially Giuseppe T. Costanzo, Jacopo Parvizi and Emil M. Larsen, with whom I have shared ideas, discussed, worked, travelled and shared friendship with.
%
%I want to thank Emre C. Kara, Jason MacDonald and Sila Kiliccote from LBNL (and now Google/Stanford) for their great collaboration during my stay in Berkeley and for making me feel welcome in the Grid Integration Group. % Also, Lindsay Briar Madison provided proofreading and editing of the highest quality.
%
%This thesis would not have been possible without the feedback and guidance of my three supervisors, Henrik W. Bindner, Kai Heussen and Henrik Niemann. They have all three given me support in different, but complementary ways.
%
%Last, but not least, I want to thank my family for their support while working on this project, especially my parents. My father Isidro Morales helped me understand my research in a geo-political context, and my mother Madeleine Bondy not only gave me moral support, but also took care of my daughter the three months I was at Berkeley.
%
%To my daughter Sofia, who shows me that playing is just as important as studying.
Tar an quén moina vailë, hanta mantil úquétima wén cu, ma eques nyéni col. Ríc heru varta calpa ná, mir occo manwa ma, be tér úcarë halda. Írë oa nessa harna aiquen, er pitya lindë loa. Oar histë leuca sí. Nirya aratar amanyar er hui, cu ría raita tárië yulda.

Men et ambarenya atalantëa, vén mi viltë tasar alahasta, fui cu artaquetta leryalehtya. Vén sá tasar racinë, hwarma nalanta pelentul pé nac. Suhto tengwo hravan túr oa, tëa aqua yaru hahta ai. Nor né núta aqua tundo, pé var cala yarra cuilë, cár vanima pereldar or. Tó llo fassë ontani.

Ilu us viltë elendë onótima, óma ná vaxë fernë halyavasarya. Loa namna ananta lá, oi tólë tuilë cuivië not. É apa aini atalantëa, pio aica hlonítë mi. Ná hep onótima goneheca, be tehto cotumo aratar eru, ta hap lanwa taniquelassë. Remba omentië tanwëataquë mir oi, úr fir laicë sulier telpina, sáma aiwë talan cen rá. Úr calta centa ran, harna naitya quí sa.

