\marginnote{This subsection is intended for readers who are not already familiar with the power system. It clarifies basic concepts like production/consumption balance, system frequency, System Operators, Energy markets, etc.}
The goal of the power system is to provide electricity to the population.
The electric power system today is composed of two layers (Figures~\ref{fig:powernow}-\ref{fig:marketnow}): 
\begin{description}
	\item[Physical grid] This is the level at which the electricity flows, going from generators to transmission system, to distribution system and finally to the end consumer.
	\item[Market layer] This is where all the energy trade and business operations are made. This includes the sale of electricity from producers to balance responsible consumers (BRCs). Retailers in turn buy electricity from the BRCs and sell it to the end consumer. Being a Balance Responsible Party (BRP), either as a consumer or as a producer, means that the actor is responsible for its own forecasts and must ensure as best as possible that the actual production/consumption follows the planned schedules.
\end{description}

\begin{figure}[t]
	\centering
	\caption{The Electric Power System as seen today. The power generated is first transmitted at high voltage levels to substations, from where it is distributed at medium/low voltage to medium size consumers and households.}\label{fig:powernow}
	\includegraphics[width=\textwidth]{intro/traditional_grid_new.eps}
\end{figure}

While market regulations can be adjusted or completely changed in order to cope with the large influx of renewable energy, the physical laws cannot.
When electricity is produced it must also be consumed. With current technology it is unfeasible to store electricity in large quantities, therefore electricity companies must make forecasts on how much electricity we (consumers) are going to need next day, then buy electricity accordingly. In other words, the production of electricity must match the consumption of electricity. If there is too much electricity flowing into the system (production exceeds consumption), the system frequency increases\sidenote{The system frequency is a measure of the balance of the grid. Electricity is traditionally produced with turbines which rotate at a given frequency, 50 Hz in Europe, and all the generators in an area rotate synchronously.}%\todo{Check up on sources}}
, and might eventually damage electric components in the grid. Vice versa, too little electricity in the system (consumption exceeds production) will cause a blackout. 
\begin{figure*}[h]
	\centering
	\caption{The actors and relationships in the power market today. Note that the consumer buys electricity from a retailer, but has no further contact to the other market actors.}\label{fig:marketnow}
	\includegraphics[width=\textwidth]{intro/market_now.eps}
\end{figure*}

Forecasts will always have an error margin, so there will always be an unbalance between production and consumption of electricity. The Transmission System Operator (TSO) is the entity responsible of resolving the unbalances of the system and maintaining the correct operation of the system. In order to do this, it buys ancillary services from the generators. This market relationship is also reflected in Figure~\ref{fig:marketnow}. There are many different kinds of services, and \todo{[find a source listing the ancillary services, entsoe or sedc]} gives a thorough overview of these. Here it suffices to say\footnote{Further discussion on ancillary services will be presented in section \textcolor{red}{appropriate section}} that for most ancillary services, the TSO will pay generators to deviate from their planned production plans in order to bring the system back to balance. In the future, we expect that the way the electric power system works will change.
