%!TEX root = ../Thesis.tex
\chapter{The Aggregator} % (fold)
\label{cha:aggregator}
The concept of aggregators has become widespread in the smart grid literature. It is clear from the different uses it is given that the concept is still not clearly defined. This chapter seeks to explain what an aggregator is, and define the terminology that will be used throughout the rest of the thesis. The concepts presented here were originally present in as a work-in-progress paper\fcite{bondy2015a} which can be found in Appendix~\ref{app:etfa2015}. 

\section{Background}
In this work the concept of aggregation encompasses the creation and management of a portfolio of flexibility assets which seeks to provide the pooled flexibility as a service. It seems that this general definition covers most uses of the word in the literature but there is still a wide spread in terms of the functionality aggregators are expected to have. This can be seen by the wide variety of aggregator designs in the literature, see e.g.\cite{kok2005powermatcher,han2010development,sortomme2011optimal,costanzo2013coordination}.

In some work\fcite{fenix2009} a distinction between aggregators is made in terms of which kind of task they perform. If they provide ancillary services they are catalogued as Technical Virtual Power Plants (VPPs) and if they trade energy in the day-ahead energy market they are catalogued as Commercial VPPs. But recent work\fcite{niesse2014conjoint} proposes a \emph{Dynamic VPP}, which is an aggregator that is able to participate both in day-ahead markets and ancillary service markets. This type of advanced design could become commonplace in future, rendering moot the \emph{Commercial} vs. \emph{Technical VPP} classification. 

Other work classifies 


\section{Clarifying the Aggregator Concept}

\section{Advantages brought by the Aggregator}
Advantages of using an aggregator: statistical, scalability (the arguments from AggRefArch). What is the contribution:
\begin{itemize}
	\item Aggregator Reference Architecture
	\item What can we test, and what should be tested?
	\item In lesser scale, contributed to the DMPC
\end{itemize}

Aggregators provide two kinds of services\footnote{The terminology used in papers [aggrefarch,isgt2014] vary a bit, but we have settled on the terminology used here}:
\begin{description}
	\item[Flexibility services] which are provided to SOs and BRPs
	\item[Asset management services] provided to the owners of the units
\end{description}

\section{The Functional Aggregator Reference Architecture}

% chapter The Aggregator (end)
