%!TEX root = ../Thesis.tex
\chapter{Verification of Aggregator Services} % (fold)
\label{cha:verification}

\begin{itemize}
	\item Why is verification an important part of validation?
	\item Performance Assessment of Aggregators providing Demand Response
\end{itemize}


Product Delivery: Performance Measurement 
Performance measurement, which is typically termed Measurement and Verification (M\&V), is the process of quantifying and validating the provision of the service according to the specifications of product. The performance measurement process usually occurs at three stages: 
\begin{itemize}
	\item To qualify potential resources against product specifications as an entry gate to participation 
	\item To verify resource conformance to the product specifications during and after participation 
	\item To calculate the amount of product delivered by the resource as part of financial settlements
\end{itemize}
 
All resources should be held to the performance specifications established by the product. However – demand side and generation side communication requirements will usually need to be designed separately and made appropriate to each. Technical rules often proscribe the use of metered values to base performance and settlements.[from the SEDC report used in the DRAS paper]


% chapter Service Verification (end)

